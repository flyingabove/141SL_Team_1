\documentclass[paper=a4, fontsize=11pt]{scrartcl}
\usepackage[letterpaper, margin=0.85in]{geometry}
\usepackage[T1]{fontenc}
\usepackage{fourier}
\usepackage{ragged2e}

\usepackage[english]{babel}															
\usepackage[protrusion=true,expansion=true]{microtype}	
\usepackage{amsmath,amsfonts,amsthm} 
\usepackage[pdftex]{graphicx}	
\usepackage{url}


\usepackage{sectsty}
\allsectionsfont{\centering \normalfont\scshape}


\usepackage{fancyhdr}
\pagestyle{fancyplain}
\fancyhead{}											
\fancyfoot[L]{}											
\fancyfoot[C]{}											
\fancyfoot[C]{\thepage}									
\renewcommand{\headrulewidth}{0pt}			
\renewcommand{\footrulewidth}{0pt}				
\setlength{\headheight}{13.6pt}


\numberwithin{equation}{section}		
\numberwithin{figure}{section}			
\numberwithin{table}{section}				


\newcommand{\horrule}[1]{\rule{\linewidth}{#1}} 

\title{
		%\vspace{-1in} 	
		\usefont{OT1}{bch}{b}{n}
		\normalfont \normalsize \textsc{University of California, Los Angeles} \\ [25pt]
		\horrule{0.5pt} \\[0.4cm]
		\huge Cry, Cry, Cry \\
		\horrule{2pt} \\[0.5cm]
}
\author{
		\normalfont \normalsize Kyle Colton \\
        \normalfont \normalsize Christian Gao \\
        \normalfont \normalsize  Benjamin Hong \\
        \normalfont \normalsize Feiran Zhu\\
}
\date{}


\begin{document}
\maketitle

\includegraphics[width=\textwidth]{baby.jpg}
\newpage

\tableofcontents
\newpage
\noindent \horrule{0.5pt} 
{\textbf{\\ Objective.}}
The purpose of this study is to discuss algorithms designed to predict whether or not a baby is crying. 
{\textbf{\\ Data Collection and Procedure.}}
We were given a multitude of audio files (in WAV format) of animal, baby, and adult sounds. We analyzed the pitch of each sound using two different procedures: (1) using FOSS \textit{aubio} and conducting K-means clustering, which is an iterative classification method that we will discuss in more detail later, and support vector machine (SVM) classification and (2) using a specially designed pitch extraction algorithm to ultimately use the Baum-Welch algorithm to answer a number of questions that we will make clear throughout the report.
{\textbf{\\ Conclusion.}}
We arrived at a decent classification algorithm using K-means and SVM, but because of limitations in the data, we did not end up with an extremely successful algorithm based on HMM.  \\
\horrule{0.5pt} \\[0.4cm]
% * <bendhong@ucla.edu> 2015-06-06T23:36:04.154Z:
%
% 
%
% * <bendhong@ucla.edu> 2015-06-06T23:36:05.493Z:
%
% 
%
\section{Abstract}
% * <bendhong@ucla.edu> 2015-06-06T23:36:06.752Z:
%
% 
%
Caring for a baby requires patience and effective communication. A caretaker attends to the baby when he/she cries, but this strictly involves an auditory process. We can see how this becomes problematic for parents who are deaf or have difficulty hearing. Fortunately, because of technological advances, there are now devices that assist parents in determining whether or not their child is crying. Of course, for a device to accurately decipher noises, the algorithm must be able to distinguish between different noises (e.g., between a dog crying and a baby crying). This report discusses the algorithms we designed to predict the baby's emotional state---that is, is the baby crying, laughing, or neutral? 


\section{Questions}
\begin{enumerate}
	\item Can we design an algorithm that accurately predicts whether or not a baby is crying?
	\item Given that our algorithm works, what is its prediction rate---that is, how often does our algorithm successfully classify noises into their respective categories?
\end{enumerate}

\section{Variable Description}

\section{Statistical Methods Used}
\subsection{Hidden Markov Model}
\begin{center}
\begin{tabular}{ |c|c|c|c| } 
 \hline
       & Cry & Neutral & Laugh \\
 \hline
 Cry & 0.8 & 0.15 & 0.05 \\ 
 Neutral & 0.2 & 0.6 & 0.2 \\ 
 Laugh & 0.05 & 0.35 & 0.6 \\
 \hline
\end{tabular}
\end{center}

\centering $\downarrow$

\begin{center}
\begin{tabular}{ |c|c|c|c| } 
 \hline
       & Cry & Neutral & Laugh \\
 \hline
 Cry & 0.85 & 0.1 & 0.05 \\ 
 Neutral & 0.15 & 0.7 & 0.15 \\ 
 Laugh & 0.1 & 0.4 & 0.5 \\
 \hline
\end{tabular}
\end{center}

\justifying Let $X_t$ be a discrete hidden random variable with $N$ possible values. $P(X_t|X_{t-1})$ is independent of time $t$, so our transition matrix will be:

\begin{equation}
A = \{a_{ij}\} = P(X_t = j|X_{t-1} = i)
\end{equation}

\noindent \\ The initial state distribution at, for example, $t = 1$, is given by: $\boxed{\pi_i = P(X_1 = i)}$ \\

\noindent The probability of a certain observation occurring at time $t$ for state $j$ is given by: $\boxed{b_j(y_t) = P(Y_t = y_t|X_t = j)}$ \\

\noindent The observation sequence should look like $Y = (Y_1 = y_1, Y_2 = y_2, \ldots, Y_t = y_t)$

\noindent Altogether, we now have a hidden Markov chain that can be described by:

\begin{equation}
\boxed{\theta = (A, B, \pi)}
\end{equation}

\section{Summary of Findings}
\subsection{Using K-Means}
\subsection{Using Support Vector Machine (SVM)}
\subsection{Using Baum-Welch Algorithm}

\section{Conclusion}

\section{Shortcomings}

\section{Recommendations}

\end{document}